\NeedsTeXFormat{LaTeX2e}
\documentclass[10pt,letterpaper]{article}
\usepackage[T1]{fontenc}
\usepackage[latin9]{inputenc}
\setlength{\parskip}{\smallskipamount}
\setlength{\parindent}{0pt}
\usepackage{amssymb}
\usepackage{listings}
\usepackage{float}
%\usepackage{bm}
%\usepackage{times}
\usepackage{color}
%\usepackage[margin=0.5in]{geometry}
\usepackage{caption}
\usepackage{amsmath}
\usepackage{graphicx}
\usepackage{verbatim}

\addtolength{\voffset}{-1in}
\addtolength{\hoffset}{-1.5in}
\setlength{\topmargin}{2.5cm}
\setlength{\oddsidemargin}{2.5cm}
\setlength{\evensidemargin}{2.5cm}
\setlength{\textwidth}{16.5cm}
\setlength{\textheight}{23cm}
\setlength{\footskip}{36pt}
\setlength{\marginparsep}{0.5cm}
\setlength{\marginparwidth}{1.5cm}
\setlength{\headheight}{0pt}
\setlength{\headsep}{10pt}

\setcounter{MaxMatrixCols}{20}
\providecommand{\inv}[1]{{#1}^{\ensuremath{\mathsf{-1}}}} % inverse ...
\providecommand{\tr}[1]{{#1}^{\ensuremath{\mathsf{T}}}} % transpose ...
\providecommand{\abs}[1]{\lvert#1\rvert}
\providecommand{\norm}[1]{\left\lVert#1\right\rVert}
\providecommand{\vect}[1]{\boldsymbol#1}
\providecommand{\mat}[1]{\mathbf#1}
\providecommand{\bigTheta}{\mathit{\Theta}}
\providecommand{\bigO}{\mathit{O}}
\providecommand{\algorithmtopspace}

\title{\textbf{Kinematics} }
\author{Sam Zapolsky}
\date{}

\definecolor{light-gray}{gray}{0.95}

\lstset{
language=Matlab,
backgroundcolor=\color{white},
basicstyle=\footnotesize,
showstringspaces=false
}


\begin{document}
\maketitle
	\section*{Problem 1}
	\[		
	_s\mat{R}_w = \begin{bmatrix}\cos(\varphi^s_w) &-\sin(\varphi^s_w) \\ \sin(\varphi^s_w) & \cos(\varphi^s_w)\end{bmatrix} 
	\]
	\[		
	\vect{q}_s = \begin{bmatrix} _s\mat{R}_w \begin{bmatrix} \bar{x}_w \\ \bar{y}_w \end{bmatrix} + \vect{x}^s_w \\ \theta_w + \varphi^s_w \end{bmatrix}
	\]
	
	\[		
	_s\dot{\mat{R}}_w = \dot{\varphi}^s_w\begin{bmatrix}-\sin(\varphi^s_w) &-\cos(\varphi^s_w) \\ \cos(\varphi^s_w) & -\sin(\varphi^s_w)\end{bmatrix} 
	\]
	\[		
	\dot{\vect{q}}_s = \begin{bmatrix} _s\dot{\mat{R}}_w \begin{bmatrix} \dot{\bar{x}}_w \\ \dot{\bar{y}}_w \end{bmatrix} + \dot{\vect{x}}^s_w \\ \dot{\theta}_w + \dot{\varphi}^s_w \end{bmatrix}
	\]
	
	\[		
	_s\ddot{\mat{R}}_w = \ddot{\varphi}^s_w \begin{bmatrix}-\cos(\varphi^s_w) &-\sin(\varphi^s_w) \\ -\sin(\varphi^s_w) & \cos(\varphi^s_w)\end{bmatrix} 
	\]
	\[		
	\ddot{\vect{q}}_s = \begin{bmatrix} _s\ddot{\mat{R}}_w \begin{bmatrix} \ddot{\bar{x}}_w \\ \ddot{\bar{y}}_w \end{bmatrix} + \ddot{\vect{x}}^s_w \\ \ddot{\theta}_w + \ddot{\varphi}^s_w \end{bmatrix}
	\]

	\section*{Problem 2}
	Given:
	\[
	\vect{X}^w_s = \begin{bmatrix} -\vect{x}^s_w \\ \varphi^w_s \end{bmatrix} = \begin{bmatrix} -1 \\ 1 \\ \pi/4 \end{bmatrix}
	\]
	\[
	\dot{\vect{X}}^s_w = \begin{bmatrix} \dot{\bar{\vect{x}}}_w \\ \dot{\varphi^w_s} \end{bmatrix} = \begin{bmatrix} 0 \\ 0 \\ 1 \end{bmatrix}
	\]

	\[
	\vect{q}_w = \begin{bmatrix} \bar{\vect{x}}_w \\ \theta_w \end{bmatrix} = \begin{bmatrix} 5 \\ 5 \\ \frac{\pi}{6} \end{bmatrix}
	\]
	\[
	\dot{\vect{q}}_w = \begin{bmatrix} \dot{\bar{\vect{x}}}_w \\ \dot{\theta}_w \end{bmatrix} = \begin{bmatrix} -1 \\ 1 \\ -2 \end{bmatrix}
	\]
	\[
	\ddot{\vect{q}}_w = \begin{bmatrix} \ddot{\bar{\vect{x}}}_w \\ \ddot{\theta}_w \end{bmatrix} = \begin{bmatrix} 10 \\ 1 \\ 10 \end{bmatrix}
	\]
	compute $\dot{\vect{q}}_s$ and $\ddot{\vect{q}}_s$:

	\section*{Problem 3}
	\[
		\ddot{\vect{p}}_w = \begin{bmatrix} \ddot{\bar{x}}_w \\ \ddot{\bar{y}}_w \end{bmatrix} +  _w\ddot{\mat{R}}_i \cdot \ddot{\vect{r}}_i
	\]
	\section*{Problem 4}
		\[
		\ddot{\vect{p}}_s =\ _w\ddot{\mat{R}}_s \left( \begin{bmatrix} \ddot{\bar{x}}_w \\ \ddot{\bar{y}}_w \end{bmatrix} +  _s\ddot{\mat{R}}_i \cdot \ddot{\vect{r}}_i \right) + \ddot{\vect{x}}^s_w
		\]
	\section*{Problem 5}	
Given:
	\[
	\vect{X}^s_w = \begin{bmatrix} -\vect{x}^s_w \\ \varphi^s_w \end{bmatrix} = \begin{bmatrix} 1 \\ -1 \\ \pi/4 \end{bmatrix}
	\]
	\[
	\dot{\vect{X}}^s_w = \begin{bmatrix} \dot{\bar{\vect{x}}}^s_w \\ \dot{\varphi^s_w} \end{bmatrix} = \begin{bmatrix} \frac{\sqrt{2}}{2} \\ \frac{\sqrt{2}}{2} \\ 1 \end{bmatrix}
	\]
	\[
	\ddot{\vect{X}}^s_w = \begin{bmatrix} \ddot{\bar{\vect{x}}}^s_w \\ \ddot{\varphi^s_w} \end{bmatrix} = \begin{bmatrix} 10 \\ -10 \\ 0 \end{bmatrix}
	\]
	\[
	\vect{q}_w = \begin{bmatrix} \bar{\vect{x}}_w \\ \theta_w \end{bmatrix} = \begin{bmatrix} 5 \\ 5 \\ \frac{\pi}{6} \end{bmatrix}
	\]
	\[
	\dot{\vect{q}}_w = \begin{bmatrix} \dot{\bar{\vect{x}}}_w \\ \dot{\theta}_w \end{bmatrix} = \begin{bmatrix} -1 \\ 1 \\ -2 \end{bmatrix}
	\]
	\[
	\ddot{\vect{q}}_w = \begin{bmatrix} \ddot{\bar{\vect{x}}}_w \\ \ddot{\theta}_w \end{bmatrix} = \begin{bmatrix} 10 \\ 1 \\ 10 \end{bmatrix}
	\]
	\[
	\vect{p}_w = \begin{bmatrix} 0 \\ 0 \end{bmatrix}
	\]
	compute $\ddot{\vect{p}}_s$:
%\begin{figure}[H]
%	\begin{center}
%		\includegraphics[width=4in]{4a.png}
%		\label{default}
%	\end{center}
%\end{figure}
	
\end{document}

